\documentclass[12pt]{article}

\title{Assignment 2: CS 663, Fall 2021}
\author{\textbf{Question 2}}
\date{}
\usepackage{amsmath}
\usepackage{amssymb}
\usepackage{hyperref}
\usepackage{ulem}
\usepackage{enumitem}
\usepackage{float}
\usepackage{graphicx}
\usepackage{subcaption}
\usepackage{bm}
\usepackage[margin=0.5in]{geometry}
\begin{document}
\maketitle

\begin{itemize}
    \item Consider the two images in the homework folder `barbara256.png' and `kodak24.png'. Add zero-mean Gaussian noise with standard deviation $\sigma = 5$ to both of them. Implement a bilateral filter and show the outputs of the bilateral filter on both images for the following parameter configurations: $(\sigma_s = 2, \sigma_r = 2); (\sigma_s = 0.1, \sigma_r = 0.1); (\sigma_s = 3, \sigma_r = 15)$. Comment on your results in your report. Repeat when the image is corrupted with zero-mean Gaussian noise of $\sigma = 10$ (with the same bilaterial filter parameters). Comment on your results in your report. For the bilateral filter implementation, write a MATLAB function \texttt{mybilateralfilter.m} which takes as input an image and parameters $\sigma_r, \sigma_s$. Implement your filter using at the most two nested for-loops for traversing the image indices. For creating the filter, use functions like meshgrid and vectorization for more efficient implementation. Include all image ouputs as well as noisy images in the report. \textsf{[15 points]}
    \vspace*{0.5cm}\\
    \textbf{Answer:}
    \begin{center}
        $\mathbf{\bm{\sigma} = 5}$
    \end{center}
    % ------------------------------------------
    \textbf{Barbara256}:
    
    \begin{figure}[H]
        \centering
        \begin{minipage}{.45\textwidth}
          \centering
          \includegraphics[height=3in,width=\linewidth]{8/images/barb_orig.png}
          \caption*{Barbara original image}
          \label{fig:totalpowervst}
        \end{minipage}
        \begin{minipage}{.45\textwidth}
          \centering
          \includegraphics[height=3in,width=\linewidth]{8/images/barb_nois_5.png}
          \caption*{Barbara noisy image for $\sigma = 5$}
          \label{fig:totalpower2}
        \end{minipage}
        %\caption{}
        \label{fig:totalPower}
    \end{figure}
    
    The above noisy image is a result of adding gaussian noise of mean $0$ and standard deviation $5$.
    \newpage
    
    %-----------------------------------------------
    \textbf{Bilateral Filtering}:
    
    \begin{figure}[H]
        \centering
        \begin{minipage}{.45\textwidth}
          \centering
          \includegraphics[height=3in,width=\linewidth]{8/images/barb_flit_5_1_1.png}
          \caption*{Filtered barbara image with $\sigma_s=0.1, \ \sigma_r=0.1$}
          \label{fig:totalpowervst}
        \end{minipage}
        \begin{minipage}{.45\textwidth}
          \centering
          \includegraphics[height=3in,width=\linewidth]{8/images/barb_flit_5_2_2.png}
          \caption*{Filtered barbara image with $\sigma_s=2, \ \sigma_r=2$}
          \label{fig:totalpower2}
        \end{minipage}
        %\caption{}
        \label{fig:totalPower}
    \end{figure}
    
    \begin{figure}[H]
        \centering
        \begin{minipage}{.45\textwidth}
          \centering
          \includegraphics[height=3in,width=\linewidth]{8/images/barb_flit_5_3_15.png}
          \caption*{Filtered barbara image with $\sigma_s=3, \ \sigma_r=15$}
          \label{fig:totalpowervst}
        \end{minipage}
        %\caption{}
        %\label{fig:totalPower}
    \end{figure}
    From the above figures, we see that as we go for higher values of $\sigma_s$ and $\sigma_r$, the image gets more blurred. This is because more neighbouring pixels have significant weights in the gaussian filters. The gaussian distribution widens as the standard deviation increases.\\
    We also note that the blurring is more when $\sigma_r$ is large, $\sigma_r$ controls the extent of different intensity pixels adding up in the mean. Though bilateral filtering blurs the image, the edges are still preserved.\\[5pt]
    The same can be observed for the following kodak images.
    \newpage
    
    %------------------------------------------------------
    \textbf{Kodak26}:
    
    \begin{figure}[H]
        \centering
        \begin{minipage}{.45\textwidth}
          \centering
          \includegraphics[height=2.5in,width=\linewidth]{8/images/kod_orig.png}
          \caption*{Kodak original image}
          \label{fig:totalpowervst}
        \end{minipage}
        \begin{minipage}{.45\textwidth}
          \centering
          \includegraphics[height=2.5in,width=\linewidth]{8/images/kod_nois_5.png}
          \caption*{Kodak noisy image for $\sigma = 5$}
          \label{fig:totalpower2}
        \end{minipage}
        %\caption{}
        \label{fig:totalPower}
    \end{figure}
    
    %-----------------------------------------------
    \textbf{Bilateral Filtering}:
    
    \begin{figure}[H]
        \centering
        \begin{minipage}{.45\textwidth}
          \centering
          \includegraphics[height=2.5in,width=\linewidth]{8/images/kod_filt_5_1_1.png}
          \caption*{Filtered kodak image with $\sigma_s=0.1, \ \sigma_r=0.1$}
          \label{fig:totalpowervst}
        \end{minipage}
        \begin{minipage}{.45\textwidth}
          \centering
          \includegraphics[height=2.5in,width=\linewidth]{8/images/kod_filt_5_2_2.png}
          \caption*{Filtered barbara image with $\sigma_s=2, \ \sigma_r=2$}
          \label{fig:totalpower2}
        \end{minipage}
        %\caption{}
        \label{fig:totalPower}
    \end{figure}
    
    \begin{figure}[H]
        \centering
        \begin{minipage}{.45\textwidth}
          \centering
          \includegraphics[height=2.5in,width=\linewidth]{8/images/kod_filt_5_3_15.png}
          \caption*{Filtered kodak image with $\sigma_s=3, \ \sigma_r=15$}
          \label{fig:totalpowervst}
        \end{minipage}
        %\caption{}
        %\label{fig:totalPower}
    \end{figure}
    \newpage
    
%--------------------------------------------------------------------------
    \begin{center}
        $\mathbf{\bm{\sigma} = 10}$
    \end{center}
    
    \textbf{Barbara256}:
    
    \begin{figure}[H]
        \centering
        \begin{minipage}{.45\textwidth}
          \centering
          \includegraphics[height=3in,width=\linewidth]{8/images/barb_orig.png}
          \caption*{Barbara original image}
          \label{fig:totalpowervst}
        \end{minipage}
        \begin{minipage}{.45\textwidth}
          \centering
          \includegraphics[height=3in,width=\linewidth]{8/images/barb_nois_10.png}
          \caption*{Barbara noisy image for $\sigma = 10$}
          \label{fig:totalpower2}
        \end{minipage}
        %\caption{}
        \label{fig:totalPower}
    \end{figure}
    
    %---------------------------------------------------------------
    \textbf{Bilateral Filtering}:
    
    \begin{figure}[H]
        \centering
        \begin{minipage}{.45\textwidth}
          \centering
          \includegraphics[height=3in,width=\linewidth]{8/images/barb_flit_10_1_1.png}
          \caption*{Filtered barbara image with $\sigma_s=0.1, \ \sigma_r=0.1$}
          \label{fig:totalpowervst}
        \end{minipage}
        \begin{minipage}{.45\textwidth}
          \centering
          \includegraphics[height=3in,width=\linewidth]{8/images/barb_flit_10_2_2.png}
          \caption*{Filtered barbara image with $\sigma_s=2, \ \sigma_r=2$}
          \label{fig:totalpower2}
        \end{minipage}
        %\caption{}
        \label{fig:totalPower}
    \end{figure}
    
    \begin{figure}[H]
        \centering
        \begin{minipage}{.45\textwidth}
          \centering
          \includegraphics[height=3in,width=\linewidth]{8/images/barb_flit_10_3_15.png}
          \caption*{Filtered barbara image with $\sigma_s=3, \ \sigma_r=15$}
          \label{fig:totalpowervst}
        \end{minipage}
        %\caption{}
        %\label{fig:totalPower}
    \end{figure}
    
    %------------------------------------------------------
    \textbf{Kodak26}:
    
    \begin{figure}[H]
        \centering
        \begin{minipage}{.45\textwidth}
          \centering
          \includegraphics[height=2.5in,width=\linewidth]{8/images/kod_orig.png}
          \caption*{Kodak original image}
          \label{fig:totalpowervst}
        \end{minipage}
        \begin{minipage}{.45\textwidth}
          \centering
          \includegraphics[height=2.5in,width=\linewidth]{8/images/kod_nois_10.png}
          \caption*{Kodak noisy image for $\sigma = 10$}
          \label{fig:totalpower2}
        \end{minipage}
        %\caption{}
        \label{fig:totalPower}
    \end{figure}
    \newpage
    %-----------------------------------------------
    \textbf{Bilateral Filtering}:
    
    \begin{figure}[H]
        \centering
        \begin{minipage}{.45\textwidth}
          \centering
          \includegraphics[height=2.5in,width=\linewidth]{8/images/kod_filt_10_1_1.png}
          \caption*{Filtered kodak image with $\sigma_s=0.1, \ \sigma_r=0.1$}
          \label{fig:totalpowervst}
        \end{minipage}
        \begin{minipage}{.45\textwidth}
          \centering
          \includegraphics[height=2.5in,width=\linewidth]{8/images/kod_filt_10_2_2.png}
          \caption*{Filtered barbara image with $\sigma_s=2, \ \sigma_r=2$}
          \label{fig:totalpower2}
        \end{minipage}
        %\caption{}
        \label{fig:totalPower}
    \end{figure}
    
    \begin{figure}[H]
        \centering
        \begin{minipage}{.45\textwidth}
          \centering
          \includegraphics[height=2.5in,width=\linewidth]{8/images/kod_filt_10_3_15.png}
          \caption*{Filtered kodak image with $\sigma_s=3, \ \sigma_r=15$}
          \label{fig:totalpowervst}
        \end{minipage}
        %\caption{}
        %\label{fig:totalPower}
    \end{figure}
    As we increase the standard deviation of gaussian noise, the intensities have more error now. Hence we expect to see better filtering results than the previous one, but low $\sigma_s$ and $\sigma_r$ doesn't provide much smoothing to the noisy images. Hence we only observe smoothing for $\sigma_s=3$ and $\sigma_r=15$.
    
\end{itemize}
\end{document}